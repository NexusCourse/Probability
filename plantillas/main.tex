\documentclass[a4paper,12pt]{article}
% *** LANGUAGE PACKAGES ***
\usepackage[spanish, mexico]{babel} 
\usepackage[utf8]{inputenc}
\usepackage[T1]{fontenc}
\usepackage{lmodern}		% Great font
\renewcommand*\familydefault{\sfdefault}
\usepackage[style=spanish]{csquotes}

% *** GEOMETRY PACKAGES ***
\usepackage{geometry}
\geometry{
  a4paper,  % Cambiar a tamaño Letter
  left=30mm,
  right=30mm,
  top=40mm,
  bottom=35mm,
  headheight=40mm
}

% *** GRAPHICS RELATED PACKAGES ***
\usepackage{graphicx}       % Loading images


% *** TABLE PACKAGES ***
\usepackage{booktabs}
\usepackage{multicol}
\usepackage{enumitem}

%%% EQUATIONS %%%
\usepackage{mathtools}

% *** REFERENCES ***
\usepackage{perpage}
\MakePerPage{footnote}
\usepackage{footnote}
\usepackage[colorlinks,
citecolor=cyan,
urlcolor=blue,
linkcolor=blue,
citebordercolor={0 0 1},
urlbordercolor={0 1 1},
linktocpage,
hyperfootnotes=true
]{hyperref}

% *** COLOR PACKAGES ***
\usepackage{xcolor}		
% Color definitiions
% Definición de colores con tonos verdes
\definecolor{green1}{RGB}{34,139,34}   % Verde oscuro (Forest Green)
\definecolor{green2}{RGB}{144,238,144} % Verde claro (Light Green)
\definecolor{gray2}{RGB}{245,245,245}  % Gris muy claro como fondo (casi blanco)

%definecolor{blue2}{RGB}{10,62,157}
%\definecolor{red2}{RGB}{173,17,0}
%\definecolor{gray2}{RGB}{240,240,240}
%

%\usepackage[bindingoffset=0.0cm,vcentering=true,%
%top=2.8cm,
%bottom=3cm,
%left=2.54cm,
%right=2.54cm,
%headsep=0.75cm,
%headheight=35pt,
%footskip=2cm,
%marginparwidth=4cm,
%%showframe,
%]{geometry} %showframe para ver los margins


\usepackage{listings}
\lstset{ % Configuración general para listar código
    language={Python},                     % Lenguaje del código
    basicstyle=\small\ttfamily,            % Estilo básico del texto (monoespaciado)
    numbers= left,                          % Números de línea a la izquierda
    numberstyle=\tiny\color{black},       % Estilo de los números de línea en verde claro
    frame=tb,                              % Marco arriba y abajo
    tabsize=4,                             % Tamaño de tabulación
    columns=fixed,                         % Espaciado fijo entre columnas
    showstringspaces=false,                % No mostrar espacios en cadenas de texto
    showtabs=false,                        % No mostrar tabulaciones
    keepspaces=true,                       % Mantener los espacios en el código
    backgroundcolor=\color{gray2},         % Fondo en gris claro
    commentstyle=\color{green1}\itshape,   % Comentarios en verde oscuro e itálicas
    keywordstyle=\color{green2}\bfseries,  % Palabras clave en verde claro y negritas
    stringstyle=\color{green1},            % Cadenas de texto en verde oscuro
    breaklines = true,
    captionpos = b,
    keepspaces = true,
    showspaces = false,
    showstringspaces = false,
    showtabs = false,
    tabsize = 2
}
\usepackage{inconsolata} % Type font for \verb text
\usepackage{showexpl} % Para hacer ejemplos en LaTeX
\setlength{\parindent}{0cm}
\usepackage{xcolor}
\usepackage{amsmath}
\usepackage{amssymb}
\usepackage{booktabs}
\usepackage{tabularx}
\usepackage{tcolorbox}   % Paquete para crear cajas de colores 


% Configuración de tcolorbox
\tcolorboxenvironment{theorem}{  % Configuración para el entorno de teorema
    colback=gray!40, % Fondo gris claro
    colframe=black,  % Borde negro
    coltitle=black,  % Color del título
    boxrule=0.8pt,   % Grosor del borde
    rounded corners,   % Esquinas rectas
    fonttitle=\bfseries % Título en negritas
}

% Definición del entorno de teorema
\newtheorem{theorem}{Teorema}


% Configuración de tcolorbox para definiciones
\tcolorboxenvironment{definition}{  % Configuración para el entorno de definición
    colback=gray!10,    % Fondo gris claro
    colframe=black,     % Borde negro
    coltitle= green1,    % Color del título verde
    boxrule=0.8pt,      % Grosor del borde
    rounded corners,    % Esquinas redondeadas
    fonttitle=\bfseries % Título en negritas
}

% Definición del entorno de definición
\newtheorem{definition}{Definición}



\tcolorboxenvironment{example}{  % Configuración para el entorno de definición
    colback= white,    % Fondo gris claro
    colframe= gray,     % Borde negro
    coltitle= green1,    % Color del título verde
    boxrule=0.8pt,      % Grosor del borde
    sharp corners,    % Esquinas redondeadas
    fonttitle=\bfseries % Título en negritas
}

% Definición del entorno de definición
\newtheorem{example}{Ejercicio}

%%%%%%%%%%%%%%%%%%%%%%%%%%%%%%%%%%%%%%%%%%%%%%%%%%%%%%%%%%%%%%%%%%%%%%%%%%%%%%%%%
% CAMBIAR EL NOMBRE PRINCIPAL DE LA LECCION
\title{\textcolor{green1}{\textbf{PLANTILLA DE EJEMPLO}}}
%%%%%%%%%%%%%%%%%%%%%%%%%%%%%%%%%%%%%%%%%%%%%%%%%%%%%%%%%%%%%%%%%%%%%%%%%%%%%%%%

\author{\textit{Nexus-Probability}}
\date{}
\begin{document}
\maketitle
\begin{center}
%%%%%%%%%%%%%%%%%%%%%%%%%%%%%%%%%%%%%%%%%%%%%%%%%%%%%%%%%%%%%%%%%%%%%%%%%%%%%%%%
% CAMBIAR EL NOMBRE, NUMERO DEL CURSO Y DE LA LECCION
\textbf{CURSO \# ( Nombre del Curso)} \\ 
\vspace{0.5em}
\textit{PARTE \# / LECCIÓN \#} 
%%%%%%%%%%%%%%%%%%%%%%%%%%%%%%%%%%%%%%%%%%%%%%%%%%%%%%%%%%%%%%%%%%%%%%%%%%%%%%%%
\end{center}

\section{EJEMPLO DE PLANTILLA.}

Tenemos diferentes entornos que abarcan,\textbf{teoremas},\textbf{definiciones}, \textbf{ejemplo}, \textbf{código} y \textbf{salida de código.}

%%%%%%%%%%%%%%%%%%%%%%%%%%%%%%%%%%%%%%%%%%%%%%%%%%%%%%%%%%%%%%%%%%%%%%%%%%%%%%%%%%%%%%%
    \subsection*{Entorno Teorema}
    \begin{verbatim}
    \begin{theorem}[Aqui el nombre del teorema.]
        Aquí dentro va el teorema.
    \end{theorem}
    \end{verbatim}
        \begin{theorem}[Aqui el nombre del teorema.]
            Aquí dentro va el teorema.
        \end{theorem}
%%%%%%%%%%%%%%%%%%%%%%%%%%%%%%%%%%%%%%%%%%%%%%%%%%%%%%%%%%%%%%%%%%%%%%%%%%%%%%%%%%%%%%%

%%%%%%%%%%%%%%%%%%%%%%%%%%%%%%%%%%%%%%%%%%%%%%%%%%%%%%%%%%%%%%%%%%%%%%%%%%%%%%%%%%%%%%%
    \subsection*{Entorno Definiciones}
    \begin{verbatim}
    \begin{definition}[Aqui el nombre de la definición.]
            Aquí dentro va la definición
    \end{definition}
    \end{verbatim}
        \begin{definition}[Aqui el nombre de la definición.]
            Aquí dentro va la definición
        \end{definition}
%%%%%%%%%%%%%%%%%%%%%%%%%%%%%%%%%%%%%%%%%%%%%%%%%%%%%%%%%%%%%%%%%%%%%%%%%%%%%%%%%%%%%%%

%%%%%%%%%%%%%%%%%%%%%%%%%%%%%%%%%%%%%%%%%%%%%%%%%%%%%%%%%%%%%%%%%%%%%%%%%%%%%%%%%%%%%%%
    \subsection*{Entorno Ejemplo}
    \begin{verbatim}
    \begin{example}
            Aquí dentro el contenido del ejercicio.
    \end{example}
    \end{verbatim}
        \begin{example}
            Aquí dentro el contenido del ejercicio.
        \end{example}
%%%%%%%%%%%%%%%%%%%%%%%%%%%%%%%%%%%%%%%%%%%%%%%%%%%%%%%%%%%%%%%%%%%%%%%%%%%%%%%%%%%%%%%   

%%%%%%%%%%%%%%%%%%%%%%%%%%%%%%%%%%%%%%%%%%%%%%%%%%%%%%%%%%%%%%%%%%%%%%%%%%%%%%%%%%%%%%%
    \subsection*{Entorno Código}
        \begin{lstlisting}
        Aqui el codigo de Pyhton, copiar y pegar directo el codigo de GOOGLE COLAB
        
        def f_xy(x, y):
         if 0 < x < 1 and 0 < y < 1:
            return 6 * (x ** 2) * y
        else:
            return 0
        \end{lstlisting}
%%%%%%%%%%%%%%%%%%%%%%%%%%%%%%%%%%%%%%%%%%%%%%%%%%%%%%%%%%%%%%%%%%%%%%%%%%%%%%%%%%%%%%% 

%%%%%%%%%%%%%%%%%%%%%%%%%%%%%%%%%%%%%%%%%%%%%%%%%%%%%%%%%%%%%%%%%%%%%%%%%%%%%%%%%%%%%%%
    \subsection*{Entorno Salida de Código}
        \begin{verbatim}
        En esta parte copiar solo la salida del codigo, ya sea un valor
        numerico o tabla
        \end{verbatim}
%%%%%%%%%%%%%%%%%%%%%%%%%%%%%%%%%%%%%%%%%%%%%%%%%%%%%%%%%%%%%%%%%%%%%%%%%%%%%%%%%%%%%%%



\section{EJEMPLO EN PRACTICA}
Recordemos que la probabilidad condicional de un evento $A$ dado un evento $B$ está dada por:
\begin{equation*}
    \mathbb{P}(A | B ) = \dfrac{\mathbb{P}(A \cap B )}{\mathbb{P}(B)}
\end{equation*}

Esta definición puede extenderse al caso de funciones de probabilidad o de densidad y también para el caso de funciones de distribución. \\

    \begin{definition} [Función de Densidad de $X | Y = y$]
        Sea $(X,Y)$ un vector aleatorio discreto (o continuo) y con función de probabilidad (o de densidad) $f_{X,Y}(x,y)$. Sea $y$ un valor de la variable $Y$ tal que $f_{Y}(y) \neq 0$. A la función $x \mapsto f_{X|Y}(x,y) $ definida de la siguiente manera se le conoce como función de probabilidad (o densidad) de $X$ dado que $Y = y$
        \begin{equation}
             f_{X|Y}(x|y) = \dfrac{f_{X,Y}(x,y)}{f_{Y}(y)}
        \end{equation}
    \end{definition}

Observe que la función dada por (1) se le considera como una función de $x$ y que el valor de $y$ es fijo y que puede considerarse como un parámetro de dicha función, i.e., para cada valor fijo de $y$ se tiene una función diferente. En el caso discreto de la expresión (1) es efectivamente la definición de probabilidad condicional

    \begin{equation*}
             f_{X|Y}(x|y) = \dfrac{P(X = x, Y= y)}{P(Y=y)}
    \end{equation*}

sin embargo, recordemos que en el caso continuo las expresiones $f_{X,Y}(x,y)$ y $f_{Y}(y)$ \textbf{no son probabilidades}. Sumando o integrando sobre los posibles valores $x$, es inmediato comprobar que la función dada por (1) es efectivamente un función de probabilidad o de densidad. Observe además que cuando $X$ y $Y$ son independientes,

    \begin{equation*}
         f_{X|Y}(x|y) =  f_{X}(x)
    \end{equation*}

La formula (1) puede extenderse de manera análoga al caso de vectores de dimensión mayor. Por ejemplo, para un vector aleatorio de dimensión tres $(X,Y,Z)$ pueden calcularse funciones de densidad condicionales como $f_{X|Y,Z}(x | y,z)$ o $f_{X, Z|Y}(x, z | y)$,. Por tanto podemos dar la siguiente definición mas extendida:

    \begin{definition}[Función de Distribución Condicional.]
        Sea $(X,Y)$ un vector aleatorio con función de probabilidad o densidad $f_{X,Y}(x,y)$. Sea y un valor de $Y$ tal que $f_{Y}(y) \neq 0$. La \textbf{función de distribución condicional} de $X$ dado $Y = y$ es la función
        \begin{equation}
            f_{X|Y}(x|y) = \begin{cases} \sum_{u \leq x} f_{X,Y}(u,y) & \text{en el caso discreto, } \\ \int_{-\infty}^{x} f_{X,Y}(u,y) & \text{en el caso continuo. }  \end{cases}
        \end{equation}
    \end{definition}

De esta forma la función de distribución condicional se calcula como la suma o integral de la correspondiente función de probabilidad o densidad condicional. Nuevamente observamos que cuando $X$ y $Y$ son independientes,

    \begin{equation*}
         F_{X|Y}(x|y) =  F_{X}(x)
    \end{equation*}

\section*{Ejercicios}
    Los siguientes ejercicios propuestos tendrán solución en \textbf{Python}, por lo que te invitamos a ejecutar el código en tu computadora.

    \begin{example}
         Sea $(X,Y)$ un vector aleatorio continuo con función de densidad
            \[
            f_{X,Y}(x,y) = \begin{cases}
            6x^2y & \text{si } 0 < x < 1, 0 < y < 1, \\
            0 & \text{en otro caso.}
            \end{cases}
            \]
            Calcule las siguientes funciones:
            \begin{enumerate}[label=\alph*)]  
            \item $f_{X|Y}(x|y)$, $0 < y < 1$.
                \item $F_{X|Y}(x|y)$, $0 < y < 1$.
        \end{enumerate}
    \end{example}

    \subsection*{Solución.}
    \begin{lstlisting}
    import numpy as np
    from scipy.integrate import quad
        
    # Ejercicio 1: Calculo de densidades y funciones de distribucion
        
    # Funcion de densidad conjunta f_{X,Y}(x, y)
    def f_xy(x, y):
         if 0 < x < 1 and 0 < y < 1:
            return 6 * (x ** 2) * y
        else:
            return 0
        
    # Funcion de densidad marginal de Y
    def f_y(y):
        return quad(lambda x: f_xy(x, y), 0, 1)[0]  # Integrar respecto a x de 0 a 1
        
    # a) Calculo de f_{X|Y}(x|y)
    def f_x_given_y(x, y):
        return f_xy(x, y) / f_y(y) if f_y(y) != 0 else 0
        
    # b) Calculo de F_{X|Y}(x|y)
    def F_x_given_y(x, y):
        integral, _ = quad(lambda u: f_x_given_y(u, y), 0, x)
        return integral
    
     # Resultados para el Ejercicio 1
     y_val = 0.5  # Ejemplo para y = 0.5
     print("Ejercicio 1:")
      print(f"f_X|Y(x|{y_val}) = ", [f_x_given_y(x, y_val) for x in np.linspace(0, 1, 5)])
     print(f"F_X|Y(x|{y_val}) = ", [F_x_given_y(x, y_val) for x in np.linspace(0, 1, 5)])
    \end{lstlisting}

    \begin{verbatim}
        Ejercicio 1:
        f_X|Y(x|0.5) =  [0.0, 0.1875, 0.75, 1.6875, 0.0]
        F_X|Y(x|0.5) =  [0.0, 0.015625, 0.125, 0.421875, 1.0]
    \end{verbatim}



    \begin{example}
        Sea $(X,Y)$ un vector discreto con función de probabilidad dada por la siguiente tabla:
            \begin{table}[H]
            \centering
            \begin{tabular}{|c|c|c|c|}
            \hline
            \textbf{x/y} & 0 & 1 & 2 \\
            \hline
            0 & 0.1 & 0.05 & 0.1 \\
            \hline
            1 & 0.05 & 0.2 & 0.1 \\
            \hline
            2 & 0.05 & 0.05 & 0.3 \\
            \hline
            \end{tabular}
            \end{table}
        Calcule las siguientes funciones:
        \begin{enumerate}[label=\alph*)]  
            \item $f_{X|Y}(x|0)$
            \item $f_{X|Y}(x|1)$
            \item $F_{X|Y}(x|1)$
        \end{enumerate}
    \end{example}
    
    \subsection*{Solución.}
    \begin{lstlisting}
    # Ejercicio 2: Calculo de funciones discretas
    
    # Matriz de probabilidades P(X, Y)
    P_XY = np.array([[0.1, 0.05, 0.1],
                     [0.05, 0.2, 0.1],
                     [0.05, 0.05, 0.3]])
    
    # a) Calculo de f_X|Y(x|0)
    f_X_given_Y_0 = P_XY[:, 0] / np.sum(P_XY[:, 0])
    
    # b) Calculo de f_X|Y(x|1)
    f_X_given_Y_1 = P_XY[:, 1] / np.sum(P_XY[:, 1])
    
    # c) Calculo de F_X|Y(x|1)
    F_X_given_Y_1 = np.cumsum(f_X_given_Y_1)
    
    # Resultados para el Ejercicio 2
    print("\nEjercicio 2:")
    print("f_X|Y(x|0) =", f_X_given_Y_0)
    print("f_X|Y(x|1) =", f_X_given_Y_1)
    print("F_X|Y(x|1) =", F_X_given_Y_1)
    \end{lstlisting}

    \begin{verbatim}
        Ejercicio 2:
        f_X|Y(x|0) = [0.5  0.25 0.25]
        f_X|Y(x|1) = [0.16666667 0.66666667 0.16666667]
        F_X|Y(x|1) = [0.16666667 0.83333333 1.        ]
    \end{verbatim}
%%%%%%%%%%%%%%%%%%%%%%%%%%%%%%%%%%%%%%%%%%%%%%%%%%%%%%%%%%%%%%%%%%%%%%%%%%%%%%%%%%%%%%%%%%%%%%%%%%%%%%%%%%%%%%%%%%%%%%%%%
    
    \begin{example}
        Se lanza un dado equilibrado dos veces. Sea $X$ el resultado del primer lanzamiento y sea $Y$ el mayor de los dos resultados.
        \begin{enumerate}[label=\alph*)]  
             \item Encuentre la función de probabilidad conjunta de $X$ y $Y$.
             \item Calcule las funciones $f_{Y|X}(y | x = 3)$ y $f_{X|Y}(x | y = 3)$
        \end{enumerate}
    \end{example}

    \subsection*{Solución.}

    \begin{lstlisting}
    # Ejercicio 3: Lanzamiento de dado
    from itertools import product
    from collections import Counter
        
    # Calcular la funcion de probabilidad conjunta P(X, Y)
    lanzamientos = list(product(range(1, 7), repeat=2))
    xy_pairs = [(x, max(x, y)) for x, y in lanzamientos]
    contador = Counter(xy_pairs)
        
    # a) Funcion de probabilidad conjunta
    P_XY_dado = {k: v / len(lanzamientos) for k, v in contador.items()}
        
    # b) Calcular f_Y|X(y|x=3) y f_X|Y(x|y=3)
    def f_y_given_x(y, x):
        prob_x = sum(v for (i, j), v in P_XY_dado.items() if i == x)
        return P_XY_dado.get((x, y), 0) / prob_x if prob_x != 0 else 0
    def f_x_given_y(x, y):
        prob_y = sum(v for (i, j), v in P_XY_dado.items() if j == y)
        return P_XY_dado.get((x, y), 0) / prob_y if prob_y != 0 else 0

     # Resultados para el Ejercicio 3
     print("\nEjercicio 3:")
     print("P(X, Y) =", P_XY_dado)
     print("f_Y|X(y|x=3) =", {y: f_y_given_x(y, 3) for y in range(1, 7)})
     print("f_X|Y(x|y=3) =", {x: f_x_given_y(x, 3) for x in range(1, 7)})
    \end{lstlisting}




\end{document}

