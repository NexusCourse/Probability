% *** LANGUAGE PACKAGES ***
\usepackage[spanish, mexico]{babel} 
\usepackage[utf8]{inputenc}
\usepackage[T1]{fontenc}
\usepackage{lmodern}		% Great font
\renewcommand*\familydefault{\sfdefault}
\usepackage[style=spanish]{csquotes}

% *** GEOMETRY PACKAGES ***
\usepackage{geometry}
\geometry{
  a4paper,  % Cambiar a tamaño Letter
  left=30mm,
  right=30mm,
  top=40mm,
  bottom=35mm,
  headheight=40mm
}

% *** GRAPHICS RELATED PACKAGES ***
\usepackage{graphicx}       % Loading images


% *** TABLE PACKAGES ***
\usepackage{booktabs}
\usepackage{multicol}
\usepackage{enumitem}

%%% EQUATIONS %%%
\usepackage{mathtools}

% *** REFERENCES ***
\usepackage{perpage}
\MakePerPage{footnote}
\usepackage{footnote}
\usepackage[colorlinks,
citecolor=cyan,
urlcolor=blue,
linkcolor=blue,
citebordercolor={0 0 1},
urlbordercolor={0 1 1},
linktocpage,
hyperfootnotes=true
]{hyperref}

% *** COLOR PACKAGES ***
\usepackage{xcolor}		
% Color definitiions
% Definición de colores con tonos verdes
\definecolor{green1}{RGB}{34,139,34}   % Verde oscuro (Forest Green)
\definecolor{green2}{RGB}{144,238,144} % Verde claro (Light Green)
\definecolor{gray2}{RGB}{245,245,245}  % Gris muy claro como fondo (casi blanco)

%definecolor{blue2}{RGB}{10,62,157}
%\definecolor{red2}{RGB}{173,17,0}
%\definecolor{gray2}{RGB}{240,240,240}
%

%\usepackage[bindingoffset=0.0cm,vcentering=true,%
%top=2.8cm,
%bottom=3cm,
%left=2.54cm,
%right=2.54cm,
%headsep=0.75cm,
%headheight=35pt,
%footskip=2cm,
%marginparwidth=4cm,
%%showframe,
%]{geometry} %showframe para ver los margins


\usepackage{listings}
\lstset{ % Configuración general para listar código
    language={Python},                     % Lenguaje del código
    basicstyle=\small\ttfamily,            % Estilo básico del texto (monoespaciado)
    numbers= left,                          % Números de línea a la izquierda
    numberstyle=\tiny\color{black},       % Estilo de los números de línea en verde claro
    frame=tb,                              % Marco arriba y abajo
    tabsize=4,                             % Tamaño de tabulación
    columns=fixed,                         % Espaciado fijo entre columnas
    showstringspaces=false,                % No mostrar espacios en cadenas de texto
    showtabs=false,                        % No mostrar tabulaciones
    keepspaces=true,                       % Mantener los espacios en el código
    backgroundcolor=\color{gray2},         % Fondo en gris claro
    commentstyle=\color{green1}\itshape,   % Comentarios en verde oscuro e itálicas
    keywordstyle=\color{green2}\bfseries,  % Palabras clave en verde claro y negritas
    stringstyle=\color{green1},            % Cadenas de texto en verde oscuro
    breaklines = true,
    captionpos = b,
    keepspaces = true,
    showspaces = false,
    showstringspaces = false,
    showtabs = false,
    tabsize = 2
}
\usepackage{inconsolata} % Type font for \verb text
\usepackage{showexpl} % Para hacer ejemplos en LaTeX
\setlength{\parindent}{0cm}
\usepackage{xcolor}
\usepackage{amsmath}
\usepackage{amssymb}
\usepackage{booktabs}
\usepackage{tabularx}
\usepackage{tcolorbox}   % Paquete para crear cajas de colores 


% Configuración de tcolorbox
\tcolorboxenvironment{theorem}{  % Configuración para el entorno de teorema
    colback=gray!40, % Fondo gris claro
    colframe=black,  % Borde negro
    coltitle=black,  % Color del título
    boxrule=0.8pt,   % Grosor del borde
    rounded corners,   % Esquinas rectas
    fonttitle=\bfseries % Título en negritas
}

% Definición del entorno de teorema
\newtheorem{theorem}{Teorema}


% Configuración de tcolorbox para definiciones
\tcolorboxenvironment{definition}{  % Configuración para el entorno de definición
    colback=gray!10,    % Fondo gris claro
    colframe=black,     % Borde negro
    coltitle= green1,    % Color del título verde
    boxrule=0.8pt,      % Grosor del borde
    rounded corners,    % Esquinas redondeadas
    fonttitle=\bfseries % Título en negritas
}

% Definición del entorno de definición
\newtheorem{definition}{Definición}



\tcolorboxenvironment{example}{  % Configuración para el entorno de definición
    colback= white,    % Fondo gris claro
    colframe= gray,     % Borde negro
    coltitle= green1,    % Color del título verde
    boxrule=0.8pt,      % Grosor del borde
    sharp corners,    % Esquinas redondeadas
    fonttitle=\bfseries % Título en negritas
}

% Definición del entorno de definición
\newtheorem{example}{Ejercicio}
